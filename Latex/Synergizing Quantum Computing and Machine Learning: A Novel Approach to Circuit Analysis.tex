\documentclass{article}
\usepackage[utf8]{inputenc}
\usepackage{amsmath}
\usepackage{graphicx}
\usepackage{hyperref}
\usepackage{multicol}
\usepackage[left=1in,right=1in,top=1in,bottom=1in]{geometry} % Set margins
\title{A Hybrid Approach to Quantum Circuit Analysis: Leveraging Machine Learning for Error Rate Prediction}

\author{
Hoang Nguyen$^1$ \\
{\small $^1$\textit{Department of Physics, Université de Bourgogne, Dijon, France}}
}


\date{\today}

\begin{document}

\maketitle

\renewcommand{\thefootnote}{\arabic{footnote}}
\setcounter{footnote}{0}

\footnotetext[1]{Long-Hoang\_Nguyen@etu.u-bourgogne.fr}

\begin{abstract}
        This paper introduces a methodology that integrates quantum computing with machine learning to examine a parameterized quantum circuit with three qubits. Initiated by a Hadamard gate, the circuit progresses through RX, RZ, and RY gates, modulated by parameters $\theta$, $\phi$, and $\zeta$, and is linked by controlled-NOT gates, culminating in a measurement phase. The study leverages a dataset of various parameter configurations and corresponding output states to analyze the quantum circuit's error rates using a machine learning model. This model, composed of multiple dense layers, aims to minimize mean squared error loss. The model is trained over 200 epochs to predict error rates based on the input parameters ($\theta$, $\phi$, $\zeta$). Training progress is monitored through plots that demonstrate the model's learning curve. To validate the model's predictive capabilities, actual error rates are compared against model predictions, showing a high degree of accuracy.
\end{abstract}

\begin{multicols}{2}
\section{Introduction}
Introduce the topic of your paper, providing background information and stating the problem you intend to solve or the hypothesis you intend to test. Outline the main contributions of your paper.

\section{Quantum Circuits}
Provide an overview of quantum circuits, their significance, and their applications.

\section{Mathematical Background}
Delve into the mathematical foundation of quantum circuits. Discuss quantum bits (qubits), quantum gates, and their representation in mathematical terms. Cover the mathematical principles that govern the operations of the Hadamard gate (H), the Pauli-X (RX), the Pauli-Z (RZ), the Pauli-Y (RY) gates, and the CNOT gate (CX). Explain how these gates transform the state of qubits and the mathematical representation of these transformations.

\section{Machine Learning Algorithms}
Describe the machine learning algorithms you've used, their relevance to your research, and how they integrate with quantum circuits.

\section{Methodology}
Detail your experimental setup, the data you've used, and the procedures you've followed.

\section{Results}
Present the results of your experiments. Use figures, tables, and charts to illustrate your findings.

\section{Discussion}
Interpret your results, discussing their implications and how they relate to the hypotheses or problems stated in the introduction.

\section{Conclusion}
Summarize your findings, their significance, and potential future work.

\section*{Acknowledgments}
Acknowledge any assistance you received, including funding, advice, and technical help.

\end{multicols}

\end{document}
